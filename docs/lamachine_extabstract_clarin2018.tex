\documentclass[a4paper,11pt]{article}
%\usepackage{CLARIN2018}
% - - - - - - - IMPORTANT - - - - - - - -
% The next three lines allow XeLaTeX, graphics import and hyperlinks, set font and language
%\usepackage{xltxtra,polyglossia,graphicx,hyperref}
%\setmainfont[Mapping=tex-text]{Times}
%\setdefaultlanguage{english}
% If for some reason the above three lines are not compatible with your LaTeX installation,
% comment out the above three instructions and uncomment the following four instead:
\usepackage{times}
\usepackage{hyperref}
\usepackage{graphicx}
%\usepackage{latexsym}
\usepackage[english]{babel}

%\setlength\titlebox{5cm}

% You can expand the titlebox if you need extra space
% to show all the authors. Please do not make the titlebox
% smaller than 5cm (the original size); we will check this
% in the camera-ready version and ask you to change it back.


\title{LaMachine: A meta-distribution for NLP software}

% - - - - - - - IMPORTANT - - - - - - -
% Leave the author information empty until your paper has been accepted

% Uncomment the following line ONLY if you need two author rows
%\setlength\titlebox{80mm}

\author{Maarten van Gompel \\
  Centre for Language and Speech Technology (CLST) \\
  Radboud University Nijmegen, the Netherlands \\
  {\tt proycon@anaproy.nl} \\ %\And % if needed: this makes a second column
  \url{https://proycon.github.io/LaMachine}
%  Second Author \\
%  Department (optional)\\
%  University Name without city \\
%  City, Country \\
% {\tt email@domain} \\
% \AND % if needed: this makes a second row
%  Third Author \\
%  Department (optional)\\
%  University of City, Country \\
%  {\tt email@domain} \\\And
%  Fourth Author \\
%  Department (optional)\\
%  University Name without city \\
%  City, Country \\
%  {\tt email@domain} \\
}

\date{}

\begin{document}
\maketitle

\begin{abstract}
We introduce LaMachine, a unified Natural Language Processing (NLP) open-source software distribution to facilitate the
installation and deployment of a large amount of software projects that have been developed in the scope of the
CLARIN-NL project and its current successor CLARIAH. Special attention is paid to encouragement of good software
development practises and reuse of established infrastructure in the scientific \& open-source software development
community.
\end{abstract}

\section{Introduction} \label{intro}

Software is a key deliverable and a vital component for research in projects such as CLARIN. It is software that
provides researchers the instruments to yield for their research; without it a lot of research would become highly
unfeasible or right-out impossible. It is CLARIN's core mission to make digital language resources, including software,
available to the wider research community.

We see that NLP software often takes on complex forms such as processing pipelines invoking various individual
components, which in turn rely on various dependencies. Add dedicated web-interfaces on top of that and you obtain a
suite of interconnected software that is often non-trivial to install, configure, and deploy. This is where LaMachine
comes in.

LaMachine incorporates software providing different types of interfaces\footnote{command line interfaces, programming
interfaces, web-user interfaces, webservices} that typically address different audiences. Whilst we attempt to
accommodate both technical\footnote{data scientists, devops, system administrators, developers} and less-technical
audiences\footnote{the wider researcher community, particularly the Humanities; also educational settings}, there is a natural bias towards the former as
lower-level interfaces are often a prerequisite to build higher-level interfaces on. Depending on the \emph{flavour} of
LaMachine chosen; it makes a good virtual research environment for a data scientist, whether on a personal computer or on a computing
cluster, a good development environment for a developer and a good deployment method for production servers in for
example CLARIN centres.

We can qualify LaMachine as a \emph{meta distribution} that builds on established practises, technologies and resources.

\section{Architecture}

Being an open-source NLP software distribution, LaMachine is constrained to POSIX-compliant platforms; this primarily
means Linux, but also BSD and, with some restrictions, Mac OS X. This can be put in contrast to mobile platforms
(Android/iOS/etc), native Windows/Mac desktop software, or certain interface types in general such as classical desktop
GUI applications. Software that is incorporated must 1) bear some relevance to NLP, and 2) be under a recognised
open-source license and 3) be deposited in a public version controlled repository\footnote{e.g. Github, Gitlab,
Bitbucket}.

LaMachine is a \emph{meta distribution} as it can be installed in various contexts. At its core, LaMachine consists of a
set of machine-parsable instructions on how to obtain, install and configure software, these are implemented using
Ansible\footnote{\url{https://www.ansible.com}}.  This is notably different from the more classical notion of Linux
distributions, which generally provide their own repositories with (often binary) software packages. LaMachine builds on
this already available infrastructure by taking these repositories as a given and only needs to know which
repositories to use.  Similarly, there are different programming-language-specific ecosystems providing their own
repositories, such as the Python Package Index\footnote{\url{https://pypi.org}} for Python,
CRAN\footnote{\url{https://cran.r-project.org/}} for R, CPAN\footnote{\url{https://www.cpan.org}} for Perl, Maven
Central\footnote{\url{https://search.maven.org}} for Java.  LaMachine again relies on those to pull and install
software from and never forks, archives, or modifies the software in any way. In doing so, we compel participating
software projects to adhere to well-established distribution standards and ensure the software is more sustainable
towards the future \cite{softwarequality}. Moreover, we ensure that LaMachine never becomes a prerequisite for the software but merely a
courtesy or convenience.

LaMachine provides ample flexibility that allows it to be deployable in different contexts. First of all there is
flexibility with regard to the target platform, where we support several major GNU/Linux distributions (Debian, Ubuntu,
CentOS, RedHat Enterprise Linux, Fedora, Arch Linux), as well as Mac OS X. Second of all there is flexibility with
regard to the form, where we support \emph{containerisation} through Docker\footnote{\url{https://www.docker.com}},
\emph{virtualisation} through Vagrant\footnote{\url{https://vagrant.org}} and VirtualBox\footnote{\url{https://www.virtualbox.org}},
direct remote provisioning through Ansible (for production servers), or an installation that is either global to the
machine or local in a custom directory for a specific user. Pre-built docker containers and virtual machines with a
limited selection of participating software are regularly uploaded to the Docker Hub and Vagrant Cloud, respectively.

Installation of LaMachine begins with a single bootstrap command\footnote{See
\url{https://proycon.github.io/LaMachine}}.  It may interactively query the users for their software preferences, e.g.
the flavour of LaMachine and the set of software to install, which is never static but can be customized by the user.
The user may also opt for installing the latest releases, the more experimental development versions of the software, or
a very specific custom versions (to facilitate scientific reproducibility). The bootstrap procedure detects and installs
the necessary prerequisites automatically and eventually invokes Ansible to perform the bulk of the work.

After a successful build, the user may interact with LaMachine either through the command line, which offers a standard
shell and enables access to all lower-level tools and programming languages; or through his or her webbrowser, which
when enabled offers a simple portal page towards all installed web-capable tools. Amongst these services is also a
Jupyter Notebook environment\footnote{\url{https://jupyter.org/}}, which offers a more graphical scripting environment
for e.g. Python and R that has gained great popularity in the scientific community.

The current version of LaMachine comes with some simple data sharing facilities only: care is taken only to provide a
single shared dataspace between host and VM/container (where applicable), or allow simple upload. Extensive data
search and management functions are deliberately beyond the scope of LaMachine, and left to more high-level tooling.

With LaMachine we also attempt to harmonise the metadata of all installed software, by converting metadata upstream repositories to
the CodeMeta standard\footnote{\url{https://codemeta.github.io/}} \cite{codemeta,codemetar} where possible, or encouraging software developers to provide
their own. This in turn enables other tools to do proper service discovery, and provenance logging.

\section{Software}

As LaMachine was conceived initially as the primary means of distribution of the software stack developed at CLST,
Radboud University Nijmegen, it includes a lot of our software. A full list goes beyond the scope of this overview, we
will merely mention some CLARIN-NL/CLARIAH-funded tools: ucto (a tokeniser), Frog (an NLP suite for Dutch), FoLiA (Format
for linguistic Annotation, with assorted tools), FLAT (a web-based linguistic Annotation tool), PICCL (an OCR and
post-OCR correction pipeline) and CLAM. We already include some relevant software by Groningen University and VU
Amsterdam as well. Moreover, LaMachine incorporates a large number of renowned tools by external international parties,
offering most notably a mature Python environment\footnote{scipy, numpy, scikit-learn, matplotlib, nltk, spacy, pytorch,
keras, gensim, etc...}, but also offering space for e.g. R and Java.

\section{Conclusion \& Future work}

The very recent release of LaMachine v2, which constituted a full rewrite, has opened up LaMachine to outside
contribution. Contributor documentation has been written, and at this stage, we greatly welcome external participants
to join in. Aside from the incorporation of new relevant software, the main objectives for the future are to provide
greater \emph{interoperability} between the included tools through better \emph{high-level interfaces} for the
researcher. We see this as a bottom-up process and have now established a firm foundation to build upon. Note that such
proposed interfaces, including the current portal application in LaMachine, are always considered separate independent
software projects, which may be deployed by/in/for LaMachine, but also in other contexts. LaMachine remains `just'
a software distribution at heart.

Development of LaMachine presently takes place in the scope of the CLARIAH WP3 Virtual Research Environment (VRE)
project\footnote{https://github.com/meertensinstituut/clariah-wp3-vre}, which has much higher ambitions in accommodating
the researcher and connectivity of data and services, and transcends also those of the CLARIN LRS Switchboard
project~\cite{switchboard}. An important part of our future focus will therefore be on interoperability with the
higher-level tools emerging from the VRE efforts, but also with other parts of the CLARIN infrastructure; single-sign on
authentication being a notable example here.

\bibliographystyle{acl}
\bibliography{lamachine}



\end{document}

