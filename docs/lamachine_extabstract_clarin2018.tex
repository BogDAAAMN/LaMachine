\documentclass[a4paper,11pt]{article}
%\usepackage{CLARIN2018}
% - - - - - - - IMPORTANT - - - - - - - -
% The next three lines allow XeLaTeX, graphics import and hyperlinks, set font and language
%\usepackage{xltxtra,polyglossia,graphicx,hyperref}
%\setmainfont[Mapping=tex-text]{Times}
%\setdefaultlanguage{english}
% If for some reason the above three lines are not compatible with your LaTeX installation,
% comment out the above three instructions and uncomment the following four instead:
%\usepackage{times}
\usepackage{hyperref}
\usepackage{graphicx}
%\usepackage{latexsym}
\usepackage[english]{babel}

%\setlength\titlebox{5cm}

% You can expand the titlebox if you need extra space
% to show all the authors. Please do not make the titlebox
% smaller than 5cm (the original size); we will check this
% in the camera-ready version and ask you to change it back.


\title{LaMachine NLP Software Distribution}

% - - - - - - - IMPORTANT - - - - - - -
% Leave the author information empty until your paper has been accepted

% Uncomment the following line ONLY if you need two author rows
%\setlength\titlebox{80mm}

\author{Maarten van Gompel \\
  Centre for Language and Speech Technology \\
  Radboud University Nijmegen, the Netherlands \\
  {\tt proycon@anaproy.nl} \\ %\And % if needed: this makes a second column
%  Second Author \\
%  Department (optional)\\
%  University Name without city \\
%  City, Country \\
% {\tt email@domain} \\
% \AND % if needed: this makes a second row
%  Third Author \\
%  Department (optional)\\
%  University of City, Country \\
%  {\tt email@domain} \\\And
%  Fourth Author \\
%  Department (optional)\\
%  University Name without city \\
%  City, Country \\
%  {\tt email@domain} \\
}

\date{}

\begin{document}
\maketitle

\begin{abstract}
We introduce LaMachine, a unified Natural Language Processing (NLP) open-source software distribution to facilitate the
installation and deployment of a large amount of software projects that have been developed in the scope of the
CLARIN-NL project and its current successor CLARIAH. Special attention is paid to encouragement of good software
development practises and reuse of established infrastructure in the scientific \& open-source software development
community.
\end{abstract}

\section{Introduction} \label{intro}

Software is a key deliverable and a vital component for research in projects such as CLARIN. It is software that
provides researchers the instruments to yield for their research; without it a lot of research would become highly
unfeasible or right-out impossible. It is CLARIN's core mission to make digital language resources, including software,
available to the wider research community.

We see that NLP software often takes on complex forms such as processing pipelines invoking various individual
components. Add dedicated web-interfaces on top of that and you obtain a suite of interconnected software that is often
non-trivial to install, configure, and deploy. This is where LaMachines comes in.

LaMachine incorporates software providing different types of interfaces\footnote{command line interfaces, programming
interfaces, web-user interfaces, webservice} that typically address different audiences. Whilst we attempt to
accommodate both technical\footnote{data scientists, devops, system administrators, developers} and less-technical
audiences\footnote{end-user researchers, educational settings}, there is a natural bias towards the former as
lower-level interfaces are often a prerequisite to build higher-level interfaces on. Depending on the \emph{flavour} of
LaMachine chosen; it makes a good virtual research environment for a data scientist, whether on a personal computer or on a computing
cluster, a good development environment for a developer and a good deployment method for production servers in for
example CLARIN centres.

We can qualify LaMachine as a \emph{meta distribution} that builds on established practises, technologies and resources.

\section{Architecture}

Being an open-source NLP software distribution, LaMachine is constrained to POSIX-compliant platforms; this primarily
means Linux, but also BSD and Mac OS X, and can be contrasted to mobile platforms (Android/iOS/etc), native Windows/Mac
desktop software, or certain interface types in general such as classical desktop GUI applications. Software that is
incorporated must 1) bear some relevance to NLP, and 2) be under a recognised open-source license, and is, where
necessary, compiled from source.

As briefly stated before, we can more accurately describe LaMachine as a \emph{meta distribution}, as it can be
installed in various contexts. At its core, LaMachine consists of a set of machine-parsable instructions on how to
obtain, install and configure software, these are implemented using Ansible\footnote{\url{https://www.ansible.com}}.
This is notably different from the more classical notion of Linux distributions, which generally provide their own
repositories with (often binary) software packages. LaMachine builds on this already available infrastructure by taking
these repositories as a given already and only needs to know which repositories to use.  Similarly, there are different
programming language-specific ecosystems providing their own repositories, such as the Python Package
Index\footnote{\url{https://pypi.org}} for Python, CRAN\footnote{\url{https://cran.r-project.org/}} for R,
CPAN\footnote{\url{https://www.cpan.org}} for Perl, Maven Central\footnote{\url{https://search.maven.org}} for Java.
LaMachine again simply relies on those to pull and install software from. LaMachine does not fork, archive, or modify the software
in any way. In doing so, we compel participating software projects to adhere to well-established distribution standards
and ensure the software is more sustainable towards the future. Moreover, we ensure that LaMachine never becomes a
prerequisite for the software but merely a courtesy or convenience.

LaMachine provides ample flexibility that allows it to be deployable in different contexts. First of all there is
flexibility with regard to the target platform, where we support several major GNU/Linux distributions (Debian, Ubuntu,
CentOS, RedHat Enterprise Linux, Fedora, Arch Linux), as well as Mac OS X. Second of all there is flexibility with
regard to the form, where we support containerisation through Docker, virtualisation through
Vagrant\footnote{\url{https://vagrant.org}} and VirtualBox\footnote{\url{https://www.virtualbox.org}},
direct remote provisioning through Ansible (for production servers), or an installation that is either global to the
machine or local in a separate directory for a specific user. Pre-built docker containers and virtual machines with a
limited selection of participating software are regularly uploaded to the Docker Hub and Vagrant Cloud, respectively. We
refer to these different context forms as different \emph{flavours}.

Installation of LaMachine begins with a single bootstrap command\footnote{Just run
\verb|bash <(curl -s https://raw.githubusercontent.com/proycon/LaMachine/master/bootstrap.sh)|
in a terminal to get started} that can
interactively query the users for their software preferences, e.g. the flavour of LaMachine and the set of software to
install, which is never static but can be customized by the user. The user may also opt for installing the latest
releases, the more experimental development versions of the software, or a very specific custom versions (to facilitate
scientific reproducibility). The bootstrap procedure detects and installs the necessary prerequisites automatically and
eventually invokes Ansible to perform the bulk of the work.








\end{document}

