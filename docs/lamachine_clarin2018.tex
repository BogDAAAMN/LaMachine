\documentclass[a4paper,11pt]{article}
\usepackage{CLARIN2018}
% - - - - - - - IMPORTANT - - - - - - - -
% The next three lines allow XeLaTeX, graphics import and hyperlinks, set font and language
\usepackage{xltxtra,polyglossia,graphicx,hyperref}
\setmainfont[Mapping=tex-text]{Times}
\setdefaultlanguage{english}
% If for some reason the above three lines are not compatible with your LaTeX installation,
% comment out the above three instructions and uncomment the following four instead:
%\usepackage{times}
%\usepackage{url}
%\usepackage{latexsym}
%\usepackage[english]{babel}

%\setlength\titlebox{5cm}

% You can expand the titlebox if you need extra space
% to show all the authors. Please do not make the titlebox
% smaller than 5cm (the original size); we will check this
% in the camera-ready version and ask you to change it back.

\usepackage{covington} % if needed, for linguistic examples

\title{LaMachine NLP Software Distribution}

% - - - - - - - IMPORTANT - - - - - - -
% Leave the author information empty until your paper has been accepted

% Uncomment the following line ONLY if you need two author rows
%\setlength\titlebox{80mm}

\author{Maarten van Gompel \\
  Centre for Language and Speech Technology \\
  Radboud University Nijmegen, the Netherlands \\
  {\tt proycon@anaproy.nl} \\ %\And % if needed: this makes a second column
%  Second Author \\
%  Department (optional)\\
%  University Name without city \\
%  City, Country \\
% {\tt email@domain} \\
% \AND % if needed: this makes a second row
%  Third Author \\
%  Department (optional)\\
%  University of City, Country \\
%  {\tt email@domain} \\\And
%  Fourth Author \\
%  Department (optional)\\
%  University Name without city \\
%  City, Country \\
%  {\tt email@domain} \\
}

\date{}

\begin{document}
\maketitle

\begin{abstract}
We introduce LaMachine, a unified Natural Language Processing (NLP) open-source
software distribution to facilitate the installation and deployment of a large
amount of software projects that have been developed in the scope of the
CLARIN-NL project and its current successor CLARIAH. Special attention is paid
to encouragement of good software development practises.
\end{abstract}

\section{Introduction} \label{intro}

Software is a key deliverable and a vital component for research in projects
such as CLARIN. It is software that provides researchers the instruments to
yield for their research, and without it a lot of research would become highly
unfeasible or right-out impossible. It is CLARIN's core mission to make digital
language resources, including software, available to the wider research
community.

Software is a broad but well-ingrained notion in society nowadays, referring to
any form of computer program. This can manifest in various forms, whether it is
an app on a phone, a graphical desktop application, a command line tool,  or a
webpage.

one classification could be
made based on the \emph{interfaces} software offers.








We too often see that the research community develops valuable

