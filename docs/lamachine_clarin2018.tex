\documentclass[a4paper,11pt]{article}
%\usepackage{CLARIN2018}
% - - - - - - - IMPORTANT - - - - - - - -
% The next three lines allow XeLaTeX, graphics import and hyperlinks, set font and language
%\usepackage{xltxtra,polyglossia,graphicx,hyperref}
%\setmainfont[Mapping=tex-text]{Times}
%\setdefaultlanguage{english}
% If for some reason the above three lines are not compatible with your LaTeX installation,
% comment out the above three instructions and uncomment the following four instead:
%\usepackage{times}
\usepackage{hyperref}
\usepackage{graphicx}
%\usepackage{latexsym}
\usepackage[english]{babel}

%\setlength\titlebox{5cm}

% You can expand the titlebox if you need extra space
% to show all the authors. Please do not make the titlebox
% smaller than 5cm (the original size); we will check this
% in the camera-ready version and ask you to change it back.


\title{LaMachine NLP Software Distribution}

% - - - - - - - IMPORTANT - - - - - - -
% Leave the author information empty until your paper has been accepted

% Uncomment the following line ONLY if you need two author rows
%\setlength\titlebox{80mm}

\author{Maarten van Gompel \\
  Centre for Language and Speech Technology \\
  Radboud University Nijmegen, the Netherlands \\
  {\tt proycon@anaproy.nl} \\ %\And % if needed: this makes a second column
%  Second Author \\
%  Department (optional)\\
%  University Name without city \\
%  City, Country \\
% {\tt email@domain} \\
% \AND % if needed: this makes a second row
%  Third Author \\
%  Department (optional)\\
%  University of City, Country \\
%  {\tt email@domain} \\\And
%  Fourth Author \\
%  Department (optional)\\
%  University Name without city \\
%  City, Country \\
%  {\tt email@domain} \\
}

\date{}

\begin{document}
\maketitle

\begin{abstract}
We introduce LaMachine, a unified Natural Language Processing (NLP) open-source software distribution to facilitate the
installation and deployment of a large amount of software projects that have been developed in the scope of the
CLARIN-NL project and its current successor CLARIAH. Special attention is paid to encouragement of good software
development practises and reuse of established infrastructure in the scientific \& open-source software development
community.
\end{abstract}

\section{Introduction} \label{intro}

Software is a key deliverable and a vital component for research in projects such as CLARIN. It is software that
provides researchers the instruments to yield for their research; without it a lot of research would become highly
unfeasible or right-out impossible. It is CLARIN's core mission to make digital language resources, including software,
available to the wider research community.

Software is a broad but well-ingrained notion in society nowadays, referring to any form of computer program. This can
manifest in various forms, whether it is an app on a phone, a graphical desktop application, a command line tool, or a
web-based application. These different \emph{interfaces} generally address different \emph{audiences}; the data
scientist will feel at home at the command line and in scripting environments and uses these fairly low-level
interfaces, whereas researchers in the humanities demand higher-level interfaces. There is often a power trade-off
between lower and higher-level interfaces, with the former providing maximum flexibility at the cost of a steeper
learning curve, technical ability, and a do-it-yourself mentality. Higher level interfaces, on the other
hand, expose certain functionality of the software in an easy and accessible way, but in doing so often can not expose
the full power of the software. The cost trade-off is also apparant in the construction of the interfaces, where
high-level interfaces are typically far more costly to build.

LaMachine attemps to address both audiences and incorporates software providing different types of interfaces. This
explicitly includes web-based interfaces, both for human end-users as well as machines (i.e. webservices). Development
of LaMachine proceeds in a bottom-up fashion, which introduces a natural bias towards lower-level interfaces, as those
are often a prerequisite to build higher-level interfaces on.

We see that NLP software often takes on complex forms such as processing pipelines invoking various individual
components. Add dedicated web-interfaces on top of that and you obtain a suite of interconnected software that is often
non-trivial to install. This is precisely where LaMachine shines.

\section{Architecture}

\subsection{Scope}

LaMachine is an open-source NLP software distribution. This implies the following:

\begin{enumerate}
    \item LaMachine facilitates the installation, distribution and configuration of software. It does not fork, modify
        or appropriate the participating software in any way, nor does it provide a hosting place or repository for
        software.
    \item The software included in LaMachine must under an OSI-approved open-source license and is, where necessary,
        compiled from source during installation.
    \item The included software bears at least some relevance to the field of Natural Language Processing.
    \item LaMachine lives in an open-source ecosystem and therefore builds on POSIX-compliant platforms; this primarily
        means Linux, as well as BSD and Mac OS X. This by definition excludes certain software for different platforms, such as
        mobile platforms (Android/iOS/etc), native Windows/Mac desktop applications, or certain interface types in
        general such as classical desktop GUI applications.
\end{enumerate}

We can more accurately describe LaMachine as a \emph{meta distribution}: it can be installed in various forms, which we
will get back to shortly. At its core, LaMachine consists of a set of machine-parsable instructions on how to obtain,
install and configure software, these are implemented using Ansible\footnote{\url{https://www.ansible.com}}. This is
notably different from the more classical notion of Linux distributions, which generally provide their own repositories
with (often binary) software packages. LaMachine builds on this already available infrastructure by taking these
repositories as a given already and only needs to know which repositories to use.  Similarly, there are different
programming language-specific ecosystems providing their own repositories, such as the Python Package
Index\footnote{\url{https://pypi.org}} for Python, CRAN\footnote{\url{https://cran.r-project.org/}} for R,
CPAN\footnote{\url{https://www.cpan.org}} for Perl, Maven Central\footnote{\url{https://search.maven.org}} for Java.
LaMachine again simply relies on those to pull and install software from. In doing so, we compel participating software
projects to adhere to well-established distribution standards and ensure the software is more sustainable towards the
future.

LaMachine provides ample flexibility that allows it to be deployable in different contexts. First of all there is
flexibility with regard to the target platform, where we support several major GNU/Linux distributions (Debian, Ubuntu,
CentOS, RedHat Enterprise Linux, Fedora, Arch Linux), as well as Mac OS X. Second of all there is flexibility with
regard to the form, where we support containerisation through Docker, virtualisation through Vagrant and VirtualBox,
direct remote provisioning through Ansible (for production servers), or a installation that is either global to the
machine or local in a separate directory for a specific user.








\end{document}

